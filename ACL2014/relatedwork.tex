
\section{Related Work}

There has been many papers in last decade that have focused on 
approximate, streaming, and randomized algorithms on many NLP problems. 
Prior work on \lsh for noun clustering \cite{ravichandran05} applied 
original version of \lsh based on \plebf (PLEB) \cite{Indyk98STOC,Charikar02STOC}. 
The disadvantage of original \lsh algorithm is that it involves generating large number of permutations ($L=1000$, 
generating permutation is similar to generating more tables)
and sorting bit vectors of large width ($K=3000$). 
To address that issue, Goyal et al. \cite{goyal12Flag} proposed a new variant of PLEB that
is faster than the original \lsh algorithm but that still requires large number of permutations 
($L=1000$). In addition, their work can be seen as an implementing a special case of Andoni and Indyk's \lsh algorithm.  

The next major difference between our research and existing work is that the 
existing work deals with approximating cosine similarity by Hamming distance  
\cite{ravichandran05,vandurme-lall:2010:Short,vandurme-lall:2011:Short,goyal12Flag,moranACL13}. Moran et al. \shortcite{moranACL13} recently proposed a data-driven non-uniform bit allocation across hyperplanes that uses less bits than many existing \lsh schemes to approximate cosine similarity by Hamming distance.     
In all these existing problem settings, the goal is to minimize both false positives and negatives. However, we focus on minimizing false negatives with zero tolerance for false positives.

\cite{zadeh2013dimension} developed a distributed version of the LSH algorithm, for 
the Jaccard distance metric, that scales to very large text corpora by virtue of being implemented on a map-reduce, 
and by using clever sampling schemes in order to reduce the communication cost. Our work is for the cosine similarity metric, 
and uses bit flipping in a distributed manner to reduce the number of hash tables in LSH and hence the memory. 
